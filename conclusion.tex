\chapter*{Conclusion}\addcontentsline{toc}{chapter}{Conclusion}

In this thesis we have presented a series of works aimed to better understand black hole accretion through numerical simulation.  In \chap{scalefit} we analyzed GRB afterglow light curves with \scalefit, producing the first evidence that GRBs are viewed off-axis.  \chap{numerics} introduced GR-DISCO, a novel moving mesh GRMHD code for simulating disk-like flows.  In \chap{minidisk} we use GR-DISCO to study minidisks in accreting binary systems and identify spiral shock waves as a potentially efficient accretion mechanism.

GRBs are produced by ultra relativistic jets of plasma directed towards Earth.  Although much is unknown about the central engine, the GRB afterglow is a relatively well understood interaction between the ultrarelativistic jet and the circumburst medium.  In \chap{scalefit} we analyze 226 afterglow x-ray light curves from the \swiftXRT, about a third of all recorded GRBs from 2005 to 2012.  Our analysis uses high resolution numerical relativistic hydrodynamic simulations to simulate a blast wave propagating in the circumburst medium.  These simulations are fed into a radiative transfer code to calculate a template bank of high fidelity light curves and spectra.  The \scalefit\ package, developed for this work, performs Bayesian parameter estimation on afterglow light curves by fitting them to this template bank.  The jet opening angle $\thO$, the electron spectral index $p$, and for the first time the observer viewing angle $\thobs$ could be constrained in many bursts.  The distribution of $\thO$ is highly asymmetric with a median of 0.097 rad. The electron index $p$ has a median value of $2.3$ over the sample, consistent with previous results.  The distribution of $\thobs$ has a median of $0.57 \thO$ and only thirteen bursts in the entire sample reported $\thobs < 0.2 \thO$.  This provides the first evidence that GRBs are viewed off-axis, lowering inferred total jet energies by up to a factor of four.  This directly impacts the required power of the central engine, thought to be an accreting black hole or rapidly spinning magnetar.

To further the numerical study of black hole accretion we developed a new GRMHD extension to the moving-mesh code DISCO.  This three dimensional code solves the equations of general relativistic magnetohydrodynamics using sophisticated RIemann solvers and a novel constrained transport algorithm.  The moving mesh allows the code to take longer time steps while reducing error due to numerical diffusion.  In \chap{numerics} we demonstrate the efficacy of GR-DISCO on a number of test problems from relativistic hydrodynamics and relevant to accretion disks.  GR-DISCO is open source and freely available online.

The first use of GR-DISCO is a study of minidisks in binary black hole systems, presented in \chap{minidisks}.  The aim of this work was to better understand the dynamics and observational characteristics of minidisks so as to better inform global simulations of circumbinary accretion and searches for supermassive binary black hole systems.  We find that spiral shocks, excited in the outer minidisk by tidal forces, can propagate to the minidisk interior and efficiently drive accretion.  This allows minidisks to accrete with an effective $\alpha$-parameter $\sim 10^{-2}$ without the need for magnetic fields or other accretion mechanisms.  By ray-tracing through the black hole spacetime we are able to create a synthetic spectrum directly from the radiative losses of the accretion disk.  We find minidisks have a spectrum resembling the standard thin-disk models but with a small high energy excess due to shock dissipation near the innermost stable circular orbit.  Due to scale invariance of the calculation these results can be applied to any accretion disk in subject to tidal forces, in particular x-ray binaries.  

%The development and applications of the methods presented here are ongoing.  The ultimate goal for GR-DISCO is the production of high fidelity, long term simulations of black hole accretion disks.  These simulations, processed through a radiative transfer calculation, could generate a template bank of light curves and spectra.  This would serve as the backbone of a tool like \scalefit, able to constrain the physical parameters of observed accretion disks from a first principles calculation.

This dissertation comprises several works aimed to better understand black hole accretion in a variety of contexts.  \chap{numerics} is primarily methodological, developing a new numerical tool to simulate relativistic accretion disks.  \chap{minidisk} is a detailed a numerical study of a particular system, the minidisks which form during binary black hole accretion.  This study demonstrates the efficacy of spiral shock waves in driving accretion, an effect relevant for both minidisks and x-ray binaries.  \chap{scalefit} uses numerical simulations to develop a model useful to data analysis, providing the first constraints of the gamma-ray burst viewing angle.  Presented here, these works provide a small step towards a better unified understanding of black hole accretion.

