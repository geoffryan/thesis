\chapter*{Introduction}\addcontentsline{toc}{chapter}{Introduction}

Black holes, paradoxically, power some of the brightest objects in the sky. X-ray binaries, active galactic nuclei (AGN), tidal disruption events, and possibly(?) gamma-ray bursts all have at their core an accreting black hole powering their stunning emission.  This energy is emitted from an accretion disk: a thin gaseous flow slowly spiralling into the black hole, trading gravitational potential energy for orbital velocity and heat.  Hot gas radiates, producing a spectrum of emission from the optical to the x-ray observable by both terrestrial and space-based instruments.  

These systems all show a rich variety of temporal and spectroscopic behaviour.  Black hole x-ray binaries,  stellar mass black holes with a disk fed by a nearby star, undergo state changes between steady disk-dominated thermal emission and violent outbursts accompanied by non-thermal hard x-rays and radio jets.  AGN, supermassive black holes near galactic centers fed by nearby gas, display stochastic, noisy optical light curves, non-thermal x-rays, and occasionally large collimated jets.  Gamma-ray bursts (GRBs) are brief flashes of gamma-rays followed by a long afterglow of x-ray, optical, and radio emission.  Although connected with the deaths of massive stars, the GRB central engine is still unknown.  Leading candidates include accretion onto a newly-formed black hole or spin-down of a magnetar during stellar core collapse (CITE).

Most modelling of accretion disks is still based on the steady state Shakura-Sunyaev $\alpha$-disk, and its relativistic extension (CITE SS NT).  It is remarkable, and a testament to the authors, that this analytic model has held up so well over fifty years.  This figure (FIG 1) shows a recent fit to LMC X-3 using NT.  WOW!  This model has two limitations: assuming a steady state and the $\alpha$ prescription. Almost every known black hole accretion disk displays time variability on some scale.  Some effects, such as the quasi-periodic oscillations of the x-ray binaries, are subtle. Others, like a tidal disruption flare, are not.  Understanding the full diversity of accretion phenomena requires understanding their time dependence.  

The $\alpha$-prescription of the SS and NT models is an ad-hoc parameterization of the mechanism by which accretion disks transport angular momentum outwards, allowing the accretion disk to actually accrete.  The strength of this transport is controlled by a parameter, $\alpha$, commonly taken to be in $[0,1]$.  The details of the accretion mechanism in real accretion disks is still unknown.  In many disks it is probably a turbulent viscosity seeded by the magneto-rotational instability (MRI).  Other proposed mechanisms include radiation pressure, self-gravity of the accreting gas, and gravitational perturbations due to other large objects in the system.  Each one of these mechanisms provides a different ``effective $\alpha$'' with different dependence on the local gas parameters.

The non-linear dynamics of the accretion mechanism and the time-dependence of target objects leave numerical simulation as the best tool to understand the details of black hole accretion.  In Chapter 1 (LATEX THIS) I present DISCO-GR, a three dimensional moving mesh general relativistic magnetohydrodynamics (GRMHD) code designed specifically to tackle black hole accretion disks.  This code was adapted from DISCO, a Newtonian magnetohydrodynamics (MHD) code (CITE PAUL).  The moving mesh allows DISCO to take much larger time steps than equivalent fixed grid codes while also virtually eliminating the advection errors which can accumulate after integrating a solution over many orbits.  DISCO-GR includes a relativistic $\alpha$-prescription to compare with steady state models and a microphysics module to enable evolution with realistic equations of state.  The magnetic fields are evolved with a sophisticated constrained transport algorithm introduced in (CITE PAUL), preserving the divergence constraint to high precision and ensuring (TOO STRONG?) accurate evolution.  This allows GR-DISCO to move away from the $\alpha$ prescription and consider long time-scale evolution of realistic black hole accretion disks.

An open question in galaxy evolution is the fate of supermassive black hole binaries.  Due to hierarchical structure formation, every galaxy that has undergone a major merger must have hosted at least two supermassive black holes in its past.  These black holes migrate towards the galactic center, beginning to orbit each other in the presence of plentiful gas.  This gas may (will?) form a circumbinary disk, which will feed individual ``minidisks'' around each of the black holes.  This system should appear as an AGN, but perhaps with a distinct spectrum and time dependence.  Recent observations have detected candidate systems (PG1302), but these are in tension with the negative results from Pulsar Timing Arrays (Nanograv).

Chapter 2 presents the first use case for GR-DISCO: a study of the accretion dynamics in minidisks.  These are the sources of the brightest emission and are nearest the black holes: necessitating a detailed relativistic treatment.  Zooming in on an individual minidisk we find spiral shockwaves, excited by tidal forces of the binary companion, provide for efficient angular momentum transport and drive accretion.  The effective $\al$ for these shocks is a few $\times 10^{-2}$.  Ray traced images provide for accurate spectra off these disks.  While broadly similar to the NT spectrum we see a high energy excess corresponding to shock dissipation near the innermost stable orbit of the black hole.  Although these calculations were performed for minidisks around supermassive black holes any accretion disk in a binary should be subject to similar effects, including stellar mass x-ray binaries.


Although the GRB central engine is still unknown and beyond\footnote{barely} the range of direct simulation, the GRB afterglow is a very well-understood phenomenon: a relativistic blast wave producing synchrotron emission propagating through the circumburst medium.  In Chapter 3 I present a study of GRB afterglow light curves, with the ultimate aim of constraining the central engine properties.  Our model, ScaleFit, utilizes a bank of template light curves calculated from ray-traced high resolution relativistic hydrodynamic simulations.  Scaling relations allow us to keep the template bank small while still covering the entire afterglow parameter space.  We ran this model on over 200 Swift-XRT light curves.  We, for the first time, are able to measure the viewing angle of the jet, finding a median $\theta_{obs} \sim 0.6 \theta_{jet}$.  This has vast consequences for the central engine, as off-axis viewing can lower the required energy by a factor of four.  This puts the GRB energy within the range of magnetar models as well as black holes.




%In this thesis I present a series of work aimed at tackling black hole accretion through numerical simulation.
%
%
%
%X-ray binaries, compact objects in orbit around Roche-lobe filling stars, make up half of known x-ray point sources (CITE Wang16 Chandra survey) and contribute significantly to the galactic x-ray flux (CITE).  Active galactic nuclei (AGN), whose central engines are accretion flows onto a supermassive black hole (SMBH) (CITE?), can dominate the emission of their host galaxy and can be seen at cosmological distances (CITE SDSS?).  Gamma-ray bursts (GRBs), the most luminous explosions in the universe, are thought to be powered by accretion onto a newly formed black hole during stellar core collapse (CITE Andrew?).  
%
%All these objects share a common power source, accretion of gas onto a black hole.  When a gas parcel falls onto a black hole from infinity it loses an amount of potential energy almost equal to its rest mass.  Com
