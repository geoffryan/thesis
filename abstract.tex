%Accretion on to black holes powers some of the most luminous objects in the universe.  We use numerical methods to constrain the properties of black hole accretion through comparison to observational data and direct simulation. We constrain the jet opening angle and, for the first time, the viewing angle for gamma-ray bursts in the \swiftXRT{} catalogue by fitting afterglow light curves directly to hydrodynamic simulations.  The viewing angle has a median value of 0.57 of the jet opening angle over our sample, which lowers the beaming corrected total energy of gamma-ray bursts by a factor of 4, allowing more diverse central engine models. %GROSS
%
%Luminous accretion events typically occur via geometrically thin accretion disks. We develop a series of techniques for simulating such disks with numerical general relativistic hydrodynamics on a moving mesh, including viscous evolution and all relevant microphysics.  These techniques are applied to a study of minidisks: accretion disks around a single member of a binary black hole system. Tidally induced spiral shock waves are excited in the disk and propagate through the innermost stable circular orbit, providing a Reynolds stress that causes efficient accretion by purely hydrodynamic means and producing a radiative signature brighter in hard X-rays than the standard Novikov--Thorne model. Disk cooling is provided by a local blackbody prescription that allows the disk to evolve self-consistently to a temperature profile where hydrodynamic heating is balanced by radiative cooling. The spiral shock structure is in agreement with analytic predictions and generates accretion with an effective alpha parameter of order 0.01.  Ray-tracing disk emission from the simulations produces theoretical minidisk spectra and viewing-angle-dependent images for comparison with observations.


%Accretion on to black holes powers some of the most luminous objects in the universe.  I present a series of works constraining the properties of black hole accretion in a variety of astrophysical systems.  I fit gamma-ray burst (GRB) afterglow light curves observed with the \emph{Swift-XRT} directly to a suite of hydrodynamical simulations, constraining the jet opening angle and, for the first time, the viewing angle of these events.  This provides the first evidence that GRBs are observed off-axis, lowering the inferred total energy delivered by the central engine, thought to be an accreting black hole.
%
%Luminous accretion events typically occur via geometrically thin accretion disks.  I develop a series of techniques to simulate such disks using numerical general relativistic hydrodynamics on a moving mesh, including viscous evolution and all relevant microphysics.  I apply these techniques to a study of minidisks: accretion disks around a single member of a binary black hole system. Spiral shock waves, excited by tidal forces from the binary companion, propagate throughout the disk, causing efficient accretion by purely hydrodynamic means and producing a radiative signature brighter in hard X-rays than the standard Novikov--Thorne model.
%
%When binary black holes merge gravitational waves carry away energy and momentum, resulting in a kicked remnant black hole lighter than the two progenitors.  I perform simulations of the response of a thin accretion disk to such a merger, measuring the dissipation due to shock waves and the time-varying thermal emission of the disk.
%Finally, I present the first implementation of numerical general relativistic magneto-hydrodynamics on a moving mesh and study the development of turbulence in accretion disks due to the magneto-rotational instability.

Accretion on to black holes powers some of the most luminous objects in the universe. In this thesis I present a series of works aimed at constraining the properties of black hole accretion in a variety of astrophysical systems.  Numerical methods are vital for studying the multi-scale and non-linear physics of these systems. First I introduce {\tt DiscoGR}, the first implementation of numerical general relativistic magnetohydrodynamics on a moving mesh.  {\tt DiscoGR} is capable of efficiently and accurately simulating highly supersonic thin accretion disks, the objects responsible for many luminous accretion events.  I apply {\tt DiscoGR} to study minidisks: accretion disks around a single member of a binary black hole system.  Spiral shock waves, excited by tidal forces from the binary companion, propagate throughout the disk, causing efficient accretion by purely hydrodynamic means. The shock-driven accretion has an effective $\alpha$ parameter of a few$\times10^{-2}$, comparable with accretion driven by the magnetorotational instability.  Furthermore, shocks near the black hole contribute to a radiative signature brighter in hard x-rays than the standard Novikov--Thorne model.  Finally I present an analysis of gamma-ray burst (GRB) x-ray afterglow light curves. The analysis fits data from the \emph{Swift-XRT} directly to a suite of hydrodynamical simulations, constraining the jet opening angle and, for the first time, the viewing angle of these events.  I find typically the viewing angle $\theta_{\rm obs} = 0.57$ of the jet opening angle.  Observing a GRB off-axis can reduce the inferred energy of the central engine, thought to be a neutron star or accreting black hole, by up to a factor of four.


%We constrain the jet opening angle and, for the first time, the off-axis observer angle for gamma-ray bursts in the \swiftXRT{} catalogue by using the \scalefit{} package to fit afterglow light curves directly to hydrodynamic simulations.
%The \scalefit{} model uses scaling relations in the hydrodynamic and radiation equations to compute synthetic light curves directly from a set of high resolution two-dimensional relativistic blast wave simulations.  The data sample consists of all \swiftXRT{} afterglows from 2005 to 2012 with sufficient coverage and a known redshift, 226 bursts in total.  We find the jet half-opening angle varies widely but is commonly less than 0.1 radians.  The distribution of the electron spectral index is also broad, with a median at $2.30$. 
%We find the observer angle to have a median value of 0.57 of the jet opening angle over our sample, which has profound consequences for the predicted rate of observed jet breaks and affects the beaming corrected total energies of gamma-ray bursts.
%
%
%Newtonian simulations have demonstrated that accretion onto binary black holes produces accretion disks around each black hole (``minidisks''), fed by gas streams flowing through the circumbinary cavity from the surrounding circumbinary disk. We study the dynamics and radiation of an individual black hole minidisk using 2D hydrodynamical simulations performed with a new general relativistic version of the moving-mesh code Disco. We introduce a comoving energy variable that enables highly accurate integration of these high Mach number flows. Tidally induced spiral shock waves are excited in the disk and propagate through the innermost stable circular orbit, providing a Reynolds stress that causes efficient accretion by purely hydrodynamic means and producing a radiative signature brighter in hard X-rays than the Novikov--Thorne model. Disk cooling is provided by a local blackbody prescription that allows the disk to evolve self-consistently to a temperature profile where hydrodynamic heating is balanced by radiative cooling. We find that the spiral shock structure is in agreement with the relativistic dispersion relation for tightly wound linear waves. We measure the shock-induced dissipation and find outward angular momentum transport corresponding to an effective alpha parameter of order 0.01. We perform ray-tracing image calculations from the simulations to produce theoretical minidisk spectra and viewing-angle-dependent images for comparison with observations.
