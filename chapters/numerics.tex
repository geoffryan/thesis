\renewcommand{\chapid}{numerics}

% Chapter specific commands:

\newcommand{\LL}{\mathcal{L}}
\newcommand{\UU}{\mathcal{U}}
\newcommand{\FF}{\mathcal{F}}
\newcommand{\VV}{\mathcal{V}}
\renewcommand{\SS}{\mathcal{S}}
\newcommand{\dual}[1]{*\!\! #1}
\newcommand{\dualt}[1]{*\! #1}

% Math:

\chapter{Numerical Relativistic Hydrodynamics\chaplabel{numerics}}

This Chapter presents ongoing work with Paul Duffell and Andrew MacFadyen which will be incorporated into a future publication upon the public release of the GRMHD \Disco\ code.

%%%%%
% Abstract
%%%%%

\section{Chapter Abstract}

We formulate and review the system of equations relevant for astrophysical fluid flow, particularly in a relativistic context.  We develop numerical methods for solving these equations in the finite volume formulation, with a particular focus on schemes which allow for a moving numerical mesh.  

%%%%%
%Section 1 - Introduction
%%%%%

\section{Introduction} \sectlabel{intro}

At macroscopic scales the universe can be described as a $3+1$ dimensional Lorentzian manifold with metric tensor $g_{\mu\nu}$ and a number of interacting, dynamical fields.  These fields are governed by an action principle
\begin{equation}
	S = \int d^4x \sqrt{-g} \left( \frac{1}{16\pi} R - \frac{1}{4}F^{\mu\nu}F_{\mu\nu}  + \LL_M\right)\ . \eqlabel{action}
\end{equation}
In \eq{action} $R$ is the Ricci scalar of the metric tensor $g_{\mu\nu}$, $F_{\mu\nu} = \nabla_\mu A_\nu - \nabla_\nu A_\mu$ is the electromagnetic field strength tensor for the electromagnetic four potential $A_\mu$, and $\LL_M$ is the Lagrangian associated with all other matter fields and their interactions.  We will use $\LL_G$ and $\LL_{EM}$ to refer to the first and second terms of \eq{action} respectively.

  The matter fields, $A_\mu$, and $g_{\mu\nu}$ evolve in such a way that $\delta S = 0$.  The resulting Euler-Lagrange equations for $A_\mu$ take the form:
  \begin{align}
  	\nabla_\mu F^{\mu\nu} &= J^\mu \eqlabel{maxwell1} \ , \\
	\text{where } J^\mu &= \frac{\delta \LL_M}{\delta A_\mu}\ .
   \end{align}
   The construction of $F_{\mu\nu}$ is such that it obeys a Bianchi identity $\nabla_{[\mu}F_{\nu\sigma]} = 0$.  This is better stated in terms of the dual tensor $*F_{\mu\nu}$:
     \begin{align}
  	\nabla_\mu *\! F^{\mu\nu} &= 0 \eqlabel{maxwell2} \ , \\
	\text{where } *\! F^{\mu\nu} &= \frac{1}{2}\epsilon^{\mu\nu\sigma\lambda}F_{\sigma\lambda} \ .
   \end{align}
   The Equations \eqrefp{maxwell1} and \eqrefp{maxwell2} together are Maxwell's equations.
   
   Similarly, the Euler-Lagrange equations for $g_{\mu\nu}$ give the Einstein Field Equations:
  \begin{align}
	G_{\mu\nu} &= 8\pi T_{\mu\nu} \eqlabel{einstein} \ , \\
	\text{where } G_{\mu\nu} &= R_{\mu\nu} - \frac{1}{2}g_{\mu\nu} R \ , \\
	\text{and } T_{\mu\nu} &= \frac{-2}{\sqrt{-g}} \frac{\delta}{\delta g^{\mu\nu}}\left(\sqrt{-g}\left( \LL_M + \LL_{EM} \right)\right) \ . \eqlabel{defTmunu}
   \end{align} 
   The Einstein tensor $G^{\mu\nu}$ also obeys a Bianchi identity $\nabla_\mu G^{\mu\nu}=0$. By the Einstein Field Equations \eqrefp{einstein}, the stress-energy tensor $T^{\mu\nu}$ must obey the same identity:
   \begin{equation}
   	\nabla_\mu T^{\mu\nu} = 0 \eqlabel{consSE}\ .
   \end{equation}  
   \eq{consSE} expresses local conservation of energy and momentum.
   
   \subsection{3+1 Splitting of Spacetime}
   
   It can be very useful to split tensors into their space-like and time-like components.  This is referred to as $3+1$ splitting or the Arnott-DeWitt-Misner (ADM) form  (CITE).  We first introduce the unit time-like normal vector $n^\mu$, normal to surfaces of constant time $t$:
   \begin{equation}
   	n_\mu \propto \partial_\mu t\ , \qquad g_{\mu\nu}n^\mu n^\nu = -1\ .
   \end{equation}
   Such a vector can be written as $n_\mu = (-\alpha, 0, 0, 0)$, where the normalization factor $\alpha$ is the \emph{lapse}.  The contravariant components are $n^\mu = (1/\alpha, -\beta^i/\alpha)$ where $\beta^i$ is the \emph{shift}.  A spatial metric $\gamma_{\mu\nu}$ can be constructed by projecting $n^\mu$ out of the metric tensor.  The result is:
   \begin{equation}
   	\gamma_{\mu\nu} = n_\mu n_\nu + g_{\mu\nu}\ .
   \end{equation}
   One can decompose the metric tensor into the 3+1 quantities $\alpha$, $\beta^i$, and $\gamma_{ij}$.  The line element takes the ADM form:
   \begin{equation}
   	ds^2 = -\alpha^2 dt^2 + \gamma_{ij}\left(dx^i + \beta^idt\right)\left(dx^j + \beta^jdt\right)\ .
   \end{equation}
   The inverse metric takes the form:
   \begin{equation}
   	g^{\mu\nu} = \begin{pmatrix} -\alpha^{-2} & \alpha^{-2}\beta^j \\
							\alpha^{-2}\beta^i & \gamma^{ij} - \alpha^{-2}\beta^i \beta^j \end{pmatrix}\ .
   \end{equation}
   A \emph{spatial} vector is any vector $v^\mu$ such that $n_\mu v^\mu = 0$, or equivalently $v^0 = 0$.  A spatial vector has only three independent components: the spatial components $v^i$.  These spatial components may be raised and lowered with the spatial metric $\gamma_{ij}$.  By defining $\gamma \equiv \det \gamma_{ij}$ we can decompose the volume element:
   \begin{equation}
   	\sqrt{-g} = \alpha \sqrt{\gamma} \ .
   \end{equation}
   
\subsection{Matter Content and Hydrodynamics}

   The matter Lagrangian $\LL_M$ and its equations of motion are complicated functions of many interacting fields.  At macroscopic scales the fields can be described as a large number $N \gg 10^{23}$ of discrete interacting particles.  In this work we avoid dealing directly with the particle equations of motion and instead work in the \emph{fluid approximation}.    In this approximation the ensemble of particles is locally described by a number density $n(x)$, a four velocity $u^\mu(x)$, and a temperature $T(x)$.  Symmetries of $\LL_M$ correspond to conservation laws which must still be obeyed in the fluid description.  Of particular use is the conservation of baryon number, which requires the local number density of baryons $n_b(x)$ to obey a continuity equation:
   \begin{equation}
   	\nabla_\mu \left(n_b\ u^\mu\right) = 0 \eqlabel{contnb}\ .
   \end{equation}
  The fluid approximation reduces the description of matter to five degrees of freedom: the number density, the temperature, and the three independent components of the fluid velocity.  The continuity equation \eqrefp{contnb} plus the four equations of energy-momentum conservation \eqrefp{consSE} then provide enough constraints to solve the system so long as it is possible to express the stress-energy tensor $T^{\mu\nu}$ in terms of $n$, $T$, and $u^\mu$ alone, avoiding direct consideration of the material equations of motion.
  
  Consider a gas of particles of mass $m$ in a local rest frame. The gas has mass density $\rho = m n$, thermodynamic pressure $P(n,T)$, and internal energy density $u(n,T)$.  We define the total energy density $e = \rho + u$ and the specific energy density $\eps = u / \rho$.  Relativistic kinetic theory establishes (CITE) that In a local orthonormal frame the stress-energy tensor has components:
  \begin{equation}
  	T^{\hat{\mu}\hat{\nu}} = \begin{pmatrix} e & 0 & 0 & 0 \\
									0 & P & 0 & 0 \\
									0 & 0 & P & 0 \\
									0 & 0 & 0 & P \end{pmatrix}\ .
  \end{equation}
In this frame the metric tensor takes the form $g_{\hat{\mu}\hat{\nu}} = \text{diag}(-1,1,1,1)$ and the four velocity $u^{\hat{\mu}}  = (1,0,0,0)$.  This leads to the unique decomposition $T^{\hat{\mu}\hat{\nu}} = e u^{\hat{\mu}}u^{\hat{\nu}} + P ( u^{\hat{\mu}}u^{\hat{\nu}} + g^{\hat{\mu}\hat{\nu}})$.  This is an equation between tensors, and hence must be true in all frames. Defining the specific enthalpy $h = (e+P)/\rho = 1 + \eps + P/\rho$ we arrive at the stress-energy tensor for a perfect fluid:
\begin{equation}
	T^{\mu\nu}_{gas} = \rho h \ \! u^\mu u^\nu + P g^{\mu\nu} \eqlabel{defTgas}\ .
\end{equation}
The pressure and internal energy are functions of the local density and temperature, which must be determined from the properties of the particular fluid under consideration.  These relationships are referred to as the equation of state of the fluid.

\subsection{Electromagnetic Content and MHD}

The field strength tensor $F^{\mu\nu}$ is antisymmetric, $F^{\nu\mu} = -F^{\mu\nu}$, and has six independent components corresponding to the three components each of the electric and magnetic fields $E^\mu$ and $B^\mu$.  Decomposing $F^{\mu\nu}$ into electric and magnetic fields is a frame-dependent procedure.  Elementary electromagnetism teaches us that observers moving with relative velocities will observe different electric and magnetic fields.  The fields in the coordinate frame are defined by:
\begin{align}
	E^\mu = n_\nu F^{\mu\nu} \quad \text{and} \quad B^\mu = n_\nu \dual{F}^{\nu\mu}\ . \eqlabel{defEB}
\end{align}
The electric and magnetic fields are spatial vectors: $n_\mu E^\mu = n_\mu n_\nu F^{\mu\nu} = 0$ due to the antisymmetry of $F^{\mu\nu}$ and $\dualt{F}^{\mu\nu}$. They each have three independent components $E^i$ and $B^i$, totaling to the six components of $F^{\mu\nu}$.  The decomposition of $F^{\mu\nu}$ and $\dualt{F}^{\mu\nu}$ into electric and magnetic fields is:
\begin{align}
	F^{\mu\nu} &= n^\mu E^\nu - n^\nu E^\mu + \eps^{\mu\nu\sigma\lambda}n_\sigma B_\lambda\ , \eqlabel{FEB}\\
	\dualt{F}^{\mu\nu} &= B^\mu n^\nu - B^\nu n^\mu + \eps^{\mu\nu\sigma\lambda}n_\sigma E_\lambda\ .
\end{align}
The four velocity of an observer serves as their local time-like unit vector.  Hence this same decomposition can be done for \emph{any} observer merely by substituting their four velocity $u^\mu$ for $n^\mu$ and renaming the fields, for instance to $e^\mu$ and $b^\mu$.

  We can use \eq{defTmunu} to find the stress energy tensor for the electromagnetic field:
  \begin{equation}
  	T^{\mu\nu}_{EM} = F^{\mu\sigma} F^\nu_\sigma - \frac{1}{4}g^{\mu\nu}F^{\sigma \lambda}F_{\sigma\lambda}\ . \eqlabel{TEMF}
  \end{equation}
  In terms of $E^\mu$ and $B^\mu$ the stress tensor is:
  \begin{equation}
  	T^{\mu\nu}_{\text{EM}} = \left(n^\mu n^\nu + \frac{1}{2}g^{\mu\nu}\right)\left(E^2 + B^2\right) - E^\mu E^\nu - B^\mu B^\nu - \left(n^\mu \eps^{\nu \sigma \lambda \rho} + n^{\nu}\eps^{\mu \sigma \lam \rho}\right)n_\sig E_\lam B_\rho\ . \eqlabel{TEMEB}
  \end{equation}
  In many astrophysical situations the plasma under consideration has very little resistivity: any electric field in the fluid frame $u^\mu$ is quickly neutralized due to motion of electrons.  The length scale on which a plasma may have appreciable net charge density $q$ (and hence electric field) before this screening occurs is the \emph{Debye length} $\lam_{\text{Deb}} \propto \sqrt{T/q}$.  On scales larger than $\lam_{\text{Deb}}$ we can assume the plasma is neutral and has no electric field in the fluid frame.
  
  This is the central assumption of \emph{magnetohydrodynamics} (MHD), and we will restrict our study to astrophysical systems where it holds.  For a fluid (plasma) with four velocity $u^\mu$, we refer to the fluid frame electric and magnetic fields as $e^\mu = u_\nu F^{\mu\nu}$ and $b^\mu = u_\nu\dualt{F}^{\nu\mu}$. The fluid frame fields are not spatial vectors (in the coordinate frame), instead they are orthogonal to the four velocity: $u_\mu e^\mu = u_\mu b^\mu = 0$.  The MHD condition is $e^\mu = 0$.  This simplifies the field strength \eqrefp{FEB} and stress energy \eqrefp{TEMEB} tensors dramatically:
  \begin{align}
  	F^{\mu\nu}_{\text{MHD}} &= \eps^{\mu\nu\sig\lam}u_\sig b_\lam \ , \eqlabel{FMHD} \\
	\dualt{F}^{\mu\nu}_{\text{MHD}} &= b^\mu u^\nu - b^\nu u^\mu \ , \eqlabel{dFMHD} \\
  	T^{\mu\nu}_{\text{MHD}} &=  b^2 u^\mu u^\nu  + \frac{1}{2}g^{\mu\nu} b^2  - b^\mu b^\nu\ . \eqlabel{TMHD}
  \end{align}
    Constructing $B^\mu$ from $b^\mu$ and vice versa is straightforward using the definition \eqrefp{defEB} and the projection tensor $g_{\mu\nu} + u_\mu u_\nu$:
    \begin{align}
    	B^\mu &= w b^\mu - \alpha b^0 u^\mu \ , \\
	b^\mu &= \frac{1}{w}\left(B^\mu + u_\nu B^\nu u^\mu\right) \ ,
    \end{align}
    where $w \equiv -n_\mu u^\mu = \al u^0$ is the fluid Lorentz factor.  Two useful expressions are:
    \begin{align}
    	b^0 &= \frac{1}{\al} u_\mu B^\mu \ , \\
	b^2 &\equiv g_{\mu\nu} b^\mu b^\nu = \frac{1}{w^2}\left(B^2 + \left(u_\mu B^\mu\right)^2\right) \ .
    \end{align}
    We can now write $\dualt{F}^{\mu\nu}_{\text{MHD}}$ in terms of $B^\mu$ and $u^\mu$:
    \begin{equation}
    	\dualt{F}^{\mu\nu}_{\text{MHD}} = \frac{1}{w}\left(B^\mu u^\nu - B^\nu u^\mu\right) \ .
    \end{equation}
    And can finally write the sourceless Maxwell Equations \eqrefp{maxwell2} in MHD:
    \begin{align}
    	\nabla_\mu \left[\frac{1}{w}\left(B^\mu u^\nu - B^\nu u^\mu\right)\right] = 0\ .
    \end{align}
    Writing these equations in terms of coordinate derivatives and splitting off the temporal part, we arrive at four equations: the constraint $\nabla \cdot B = 0$ and the evolution equations for $B^i$:
    \begin{align}
    	\partial_j \left( \sqrt{\gam} B^j \right) &= 0 \ , \\
	\partial_t \left( \sqrt{\gam} B^i \right) + \partial_j \left(\sqrt{\gam}\left(B^i v^j - B^j v^i\right)\right) &= 0\ .
    \end{align}
    Lastly we can write down the combined stress-energy tensor for a magnetized perfect fluid in MHD.
    \begin{equation}
    	T^{\mu\nu} = \left(\rho h + b^2\right) u^\mu u^\nu + \left(P + \frac{1}{2}b^2\right)g^{\mu\nu} - b^\mu b^\nu \ . \eqlabel{TmunuGRMHD}
    \end{equation}
    This gives us all the tools we need to begin investigating these flows numerically.


\section{Hyperbolic Equations}

We have established that an ideal astrophysical plasma will obey the equations of general relativistic magnetohydrodynamics (GRMHD):
\begin{align}
	\nabla_\mu  \rho u^\mu &= 0\ , \eqlabel{GRMHD1}\\
	\nabla_\mu T^{\mu\nu} &= 0\ , \eqlabel{GRMHD2}\\
	\nabla_\mu \dual{F}^{\mu\nu} &= 0\ . \eqlabel{GRMHD3}
\end{align}
These equations all take the form of conservation laws and form a system of non-linear hyperbolic partial differential equations.  The numerical solution of hyperbolic systems of equations is a rich field of numerical analysis.  For our purposes we will focus on \emph{finite volume} schemes.  To do so we must first establish the integral form of Equations (\eqref{GRMHD1}-\eqref{GRMHD3}).  We begin by expressing the covariant derivatives as coordinate derivatives, which put (\eqref{GRMHD1}-\eqref{GRMHD3}) in the form:
\begin{align}
	\partial_\mu \left(\sqrt{-g}\ \! \rho u^\mu\right) &= 0\ , \\
	\partial_\mu\left( \sqrt{-g}\ \!T^{\mu}_\nu\right) &= \frac{1}{2}\sqrt{-g}\ \!T^{\alpha\beta}\partial_\nu g_{\alpha \beta}\ , \\
	\partial_\mu \left(\sqrt{-g}\ \! \dual{F}^{\mu\nu}\right) &= 0\ . \eqlabel{GRMHDcoord}
\end{align}
These equations are each in \emph{conservative} form
\begin{equation}
	\partial_t\ \!\UU + \partial_i \FF^i = \SS \ .
\end{equation}
We call $\UU$ the conserved quantity, $\FF^i$ the flux, and $\SS$ the source. The GRMHD system consists of eight such coupled equations which we can write as:
\begin{equation}
	\partial_t\ \!\UU_a + \partial_i \FF^i_a = \SS_a \ .
\end{equation}


We first consider a rectangular control volume $\VV$ spanning $x^1\in [x^1_-, x^1_+]$,  $x^2\in [x^2_-, x^2_+]$, and  $x^3\in [x^3_-, x^3_+]$.
%%%%%%
% Summary %
%%%%%%

\section{Summary}
\sectlabel{summary}


Summary

\section{Chapter Acknowledgements} \sectlabel{acknowledgements}

Thanks